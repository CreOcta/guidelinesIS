\chapter*{Словарь терминов}             % Заголовок
\addcontentsline{toc}{chapter}{Словарь терминов}  % Добавляем его в оглавление

\textbf{Открытый (исходный) текст}~---~данные (не обязательно текстовые),
передаваемые без использования криптографии.

\textbf{Шифротекст, шифрованный (закрытый) текст}~---~данные, полученные
после применения криптосистемы.

\textbf{Шифр, криптосистема}~---~совокупность заранее оговоренных способов преобразования исходного секретного сообщения с целью его защиты.%семейство обратимых преобразований открытого текста в шифрованный.

\textbf{Символ}~---~это любой знак, в том числе буква, цифра или знак
препинания.
\\
\textbf{Алфавит}~---~конечное множество используемых для кодирования
информации символов. Стандартный алфавит может быть изменён или дополнен
символами.
\\
\textbf{Ключ}~---~параметр шифра, определяющий выбор конкретного
преобразования данного текста. В современных шифрах криптографическая
стойкость шифра целиком определяется секретностью ключа (принцип Керкгоффса).

\textbf{Шифрование}~---~процесс нормального применения криптографического
преобразования открытого текста на основе алгоритма и ключа, в результате
которого возникает шифрованный текст.

\textbf{Расшифровывание}~---~процесс нормального применения
криптографического преобразования шифрованного текста в открытый.

\textbf{Асимметричный шифр, двухключевой шифр, шифр с открытым
ключом}~---~шифр, в котором используются два ключа, шифрующий и
расшифровывающий. При этом, зная лишь ключ зашифровывания, нельзя
расшифровать сообщение, и наоборот.

\textbf{Открытый ключ}~---~тот из двух ключей асимметричной системы, который
свободно распространяется. Шифрующий для секретной переписки и
расшифровывающий — для электронной подписи.

\textbf{Секретный ключ, закрытый ключ}~---~тот из двух ключей асимметричной
системы, который хранится в секрете. Криптоанализ — наука, изучающая
математические методы нарушения конфиденциальности и целостности информации.

\textbf{Система шифрования (шифрсистема)}~---~это любая система, которую
можно использовать для обратимого изменения текста сообщения с целью сделать
его непонятным для всех, кроме адресата.

\textbf{Криптостойкостью}~---~ это характеристика шифра, определяющая его
стойкость к дешифрованию без знания ключа (т.е. способность противостоять
криптоанализу).

\textbf{Криптоаналитик}~---~учёный, создающий и применяющий методы
криптоанализа. Криптография и криптоанализ составляют криптологию, как единую
науку о создании и взломе шифров (такое деление привнесено с запада, до этого
в СССР и России не применялось специального деления).

\textbf{Криптографическая атака}~---~попытка криптоаналитика вызвать
отклонения в атакуемой защищённой системе обмена информацией. Успешную
криптографическую атаку называют взлом или вскрытие.

\textbf{Дешифрование (дешифровка)}~---~процесс извлечения открытого текста
без знания криптографического ключа на основе известного шифрованного. Термин
дешифрование обычно применяют по отношению к процессу криптоанализа
шифротекста (криптоанализ сам по себе, вообще говоря, может заключаться и в
анализе криптосистемы, а не только зашифрованного ею открытого сообщения).

\textbf{Криптографическая стойкость}~---~способность криптографического
алгоритма противостоять криптоанализу.

\textbf{Имитозащита}~---~защита от навязывания ложной информации. Другими
словами, текст остаётся открытым, но появляется возможность проверить, что
его не изменяли ни случайно, ни намеренно. Имитозащита достигается обычно за
счет включения в пакет передаваемых данных имитовставки.

\textbf{Имитовставка}~---~блок информации, применяемый для имитозащиты,
зависящий от ключа и данных.

\textbf{Электронная цифровая подпись(электронная подпись)}~---~асимметричная
имитовставка (ключ защиты отличается от ключа проверки). Другими словами,
такая имитовставка, которую проверяющий не может подделать.

\textbf{Центр сертификации}~---~сторона, чья честность неоспорима, а открытый
ключ широко известен. Электронная подпись центра сертификации подтверждает
подлинность открытого ключа.

\textbf{Хеш-функция}~---~функция, которая преобразует сообщение произвольной
длины в число («свёртку») фиксированной длины. Для криптографической
хеш-функции (в отличие от хеш-функции общего назначения) сложно вычислить
обратную и даже найти два сообщения с общей хеш-функцией.
