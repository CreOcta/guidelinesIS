\chapter{Работа с PKI} \label{chapt2}%
\textbf{Мета роботи:~}%
Изучить методы генерации простых чисел, проверку на простоту числа и
реализовать алгоритм ДХМ.
\section{Теоретические ведомости} \label{sect2_a}
% Лабораторная 2
Подробнее о асимметричных криптосистемах.

\paragraph{Методы генерации простых чисел}

\paragraph{Известные на алгоритмы}

\paragraph{Алгоритма ДХМ}
История, цель и важность

\subparagraph{Принцип работы алгоритма}

\subparagraph{Пример реализации}
блок схема(как там мороки много)

\section{Задания}\label{sect2_b}
%
Стандартный алгоритм для выполнения работы --- Алгоритм ДХМ. Не запрещено
использовать более новые алгоритмы, соответствующие тем же принципам что и
данный ДХМ.

Вариант задания указан в \todo{табл.}.

\noindent Уровень 1.

\begin{enumerate}
    \item Описать последовательность выполнения алгоритма для шифрования
        задания по варианту указаному выше.
  \item Исследовать использование алгоритма в различных сферах ИБ. Указать
      реальное применение данного алгоритма.
\end{enumerate}

\noindent Уровень 2.

\begin{enumerate}
  \item Составить программу для реализации алгоритма ДХМ.
  \item Произвести поэтапный обмен между двумя клиентами.
  \item Предоставить все промежуточные данные, как на \todo{табл.}
  \item Продемонстрировать шифрованный блок и записать его в отчёт.
\end{enumerate}

\noindent Уровень 3.

\begin{enumerate}
  \item Разработать приложение клиент-клиент обмена информацией между
      несколькими людьми. Используя асинхронный алгоритм шифрования.
  \item Продемонстрировать выполнение приложения.
  \item Предоставить исходный код в дополнение отчёта.
\end{enumerate}
Приложение может быть составлено в любом варианте, например: онлайн передача,
запись в шифрованный файл, почтовый шифр и другие.
\section{Пример выполнения работы}\label{sect2_c}
%
\section{Варианты}\label{sect2_d}
%
\section{Вопросы для контроля}\label{sect2_e}
\begin{enumerate}
  \item Асимметричные системы.
  \item Привести примеры асимметричных алгоритмов.
  \item Криптосистемы. Использование асимметричный алгоритмов.
  \item Преимущества и недостатки асимметричный систем.
  \item Методы подбора ключа в асимметричных криптосистемах.
  \item Сферы использование асимметричных алгоритмов.
  \item Описать однонаправленные функции. Привести примеры.
\end{enumerate}
%
