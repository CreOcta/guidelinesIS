\chapter{Архивация данных} \label{chapt8}%
\textbf{Мета роботи:~}%
Выполнить исследование алгоритмов архивации данных. Использовать алгоритм
Хаффмана и Лемпеля-Зива для архивации. Восстановить данные после
\section{Теоретические ведомости} \label{sect8_a}
% Лабораторная 9
\paragraph{Архивация данных}

\paragraph{Алгоритмы архивации данных}

\subparagraph{Алгоритм Хаффмана}

\subparagraph{Алгоритм Лемпеля-Зива}

\paragraph{Актуальность архивации}

\paragraph{Восстановление данных}

\section{Задания}\label{sect8_b}
%
\begin{enumerate}
  \item Взять данные соответственно варианту из \todo{табл.}
  \item Удалить лишнюю информацию методом Хаффмана.
  \item Провести операцию методом Лемпеля-Зива.
  \item Сравнить результаты проведённых операций.
  \item Описать актуальность архивации.
  \item Сделать выводы по применению методов сжатия в различных
      криптосистемах.
\end{enumerate}
\section{Пример выполнения работы}\label{sect8_c}
%
\section{Варианты}\label{sect8_d}
%
\section{Вопросы для контроля}\label{sect8_e}
%
\begin{enumerate}
  \item Что такое архивация данных?
  \item Цель архивации?
  \item Какие Вы знаете методы архивации?
  \item Опишите принцип дерева Хаффмана.
  \item Опишите алгоритм LZ77 или его аналог.
  \item Сферы применения заданных алгоритмов.
  \item Как выбрать алгоритм, если данные заранее известны?
\end{enumerate}
