\chapter{Электронно-цифровая подпись} \label{chapt4}%
\textbf{Мета роботи:~}%
Принцип работы, особенности, создание и использование ЭЦП.
\section{Теоретические ведомости} \label{sect4_a}
% https://www.youtube.com/watch?v=nrW_v8OBOno
% https://habrahabr.ru/post/194664/
% https://tools.ietf.org/html/rfc5280


Что такое ЭЦП. (электронная подпись)

\paragraph{Структура сертификата}
%http://rsdn.org/article/ASN/ASN.xml
\paragraph{Преимущества ЭЦП}

\paragraph{Недостатки ЭЦП}

\paragraph{Применение ЭЦП}
(Интернет, https)алгоритмы и т.д.

\subparagraph{Текущие требования к стойкости ЭЦП}

\paragraph{Проверка подлинности ЭЦП}
% http://nalog-nalog.ru/spravochnaya_informaciya/kak_proishodit_proverka_podlinnosti_ecp/
% http://referatwork.ru/category/tehnologii/view/475258_identifikaciya_i_proverka_podlinnosti
\section{Задания}\label{sect4_b}

Уровень 1. %

\begin{enumerate}
  \item Выбрать и отобразить реестр сертификатов на Вашем устройстве.
  \item Опишите состав сертификата.
  \item Выберите шифрованный блок данных.
  \item Опишите полный путь данного сертификата к \todo{main-центру раздачи
      разрешений}.
\end{enumerate}

Уровень 2. %

\begin{enumerate}
  \item Создайте самоподписной сертификат.
  \item Внесите его в свой реестр.
  \item Приложите данные в отчёт.
\end{enumerate}

Уровень 3.%

\begin{enumerate}
  \item Напишите алгоритм для создания подписанного сертификата.
  \item Создайте сертификат подписанный ЭЦП из прошлого задания.
  \item Приложите пример работы программы и подписанный сертификат.
\end{enumerate}

\section{Пример выполнения работы}\label{sect4_c}
%
Пример будет включать создание ЭПЦ как самостоятельно, так и используя ПО.
\section{Варианты}\label{sect4_d}
%
\section{Вопросы для контроля}\label{sect4_e}
%
\begin{enumerate}
  \item Что такое электронно-цифровая подпись?
  \item Каков принцип работы ЭЦП?
  \item Почему ЭЦП используется в большинстве систем проверки документации?
  \item Опишите этапы шифрования-дешифровки.
  \item Как подпись обеспечивает целостность данных?
  \item Случаи небезопасного использования ЭПЦ?
  \item Где на компьютере могут храниться цифровые подписи?
  \item Как проверить надёжность ЭПЦ?
\end{enumerate}
