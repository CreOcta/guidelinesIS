\chapter{Методы атаки. Частотная атака} \label{chapt3}%
\textbf{Мета роботи:~}%
Изучение способов атаки на разные уровни системы. Методы подбора и анализ
частотной атаки.
\section{Теоретические ведомости} \label{sect2_a}
%
\textbf{Моноалфавитный подстановочный шифр} -- шифр, в котором каждой букве
исходного алфавита поставлена в соответствие одна буква шифра.


Например, возьмем слово \texttt{КУКУРУЗА}. Пусть букве \texttt{К} текста
соответствует буква \texttt{А} шифра, букве \texttt{У} текста соответствует
буква \texttt{Б} шифра, букве \texttt{Р} текста соответствует буква
\texttt{В} шифра, букве \texttt{З} текста соответствует буква \texttt{Г}
шифра, букве \texttt{А} текста соответствует буква \texttt{Д} шифра. После
подстановки букв шифра вместо букв исходного теста слово \texttt{КУКУРУЗА} в
зашифрованном виде будет выглядеть как \texttt{АБАБВБГД}. Недостатком
подобного шифрования является то, что, если какая-то буква встречается в
исходном тексте чаще всего (например, буква \texttt{О} в русском алфавите),
то и соответствующая ей буква шифра в зашифрованном тексте также встречается
чаще всего. В ниже приведенной таблице приведены частоты встречаемости букв в
английском тексте (в процентах):

\todo{диаграмма частот использования букв алфавита}

Зная частоты наиболее встречающихся букв и подсчитав, какие буквы чаще всего
встречаются в шифровке, криптоаналитик может подобрать расшифровку для
некоторых букв текста. Затем, анализируя короткие слова, найти еще буквы,
истинные значения которых можно с высокой степенью уверенности предугадать.
Например, если уже расшифрована    буква    \texttt{О}    и    в    тексте
есть слово    \texttt{ОЫО}    (подчеркнуты  уже расшифрованные буквы), то,
скорее всего, шифру \texttt{Ы} соответствует буква \texttt{Н} в исходном
тексте (\texttt{ОНО}). Чем дальше расшифровывается текст, тем легче идет
процесс расшифровки.


\section{Задания}\label{sect2_b}
%
\begin{enumerate}
  \item Освоить теорию и принципы частотной атаки.
  \item Проанализировать представленное ПО
  \item Расшифровать текст и предоставить: %
  \begin{itemize}
    \item шифрованное сообщение;
    \item перечень замен;
    \item расшифрованный текст;
    \item предоставить алгоритм дешифровки\footnote{Задание для
        дополнительных баллов.}.
  \end{itemize}
  \item Выводы к работе
\end{enumerate}
\section{Пример выполнения работы}\label{sect2_c}
%
\section{Варианты}\label{sect2_d}
%
\section{Вопросы для контроля}\label{sect2_e}
%
