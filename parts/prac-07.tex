\chapter{Сокрытие информации} \label{chapt7}%
\textbf{Мета роботи:~}%
Изучить методы, факторы и риски при сокрытии информации.
\section{Теоретические ведомости} \label{sect6_a}
%
\paragraph{Методы сокрытия информации}

\subparagraph{Использование шума}

\paragraph{Методы обнаружения информации}
 Обнаружения информации в файлах.

\section{Задания}\label{sect6_b}
Цель практической части работы состоит в получении \todo{максимально коэф.}
сокрытия информации.
%
\begin{enumerate}
  \item Изучить теорию, быть готовым к опросу.
  \item Сокрыть информацию с помощью предоставленного ПО: %
  \begin{enumerate}
    \item в тесте;
    \item в изображении;
    \item в музыке.
  \end{enumerate}
  \item Сравнение методов и выводы к работе.
\end{enumerate}
\section{Инструкция к работе с ПО}\label{sect6_c}
%
(ф-ции программы, методы и т.д.)
\section{Вопросы для контроля}\label{sect6_e}
%
\begin{enumerate}
  \item Какие есть способы сокрытия информации?
  \item В каких файлах лучше скрывать информацию?
  \item Что такое шум?
  \item Риски потери и дешифровка информации.
\end{enumerate}
