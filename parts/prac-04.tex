\chapter{Построение концепции информационной безопасности предприятия} \label{chapt4}%
\textbf{Мета роботи:~}%
Знакомство с основными принципами построения концепции ИБ предприятия, с
учётом особенностей его информационной инфраструктуры.
\section{Теоретические ведомости} \label{sect3_a}
% lab4.pdf
%https://habrahabr.ru/company/trinion/blog/322832/
%mu_ib.doc Практическая работа № 3
%https://studfiles.net/preview/5852002/page:2/
% http://securitypolicy.ru/%D1%88%D0%B0%D0%B1%D0%BB%D0%BE%D0%BD%D1%8B/%D0%BA%D0%BE%D0%BD%D1%86%D0%B5%D0%BF%D1%86%D0%B8%D1%8F_%D0%BE%D0%B1%D0%B5%D1%81%D0%BF%D0%B5%D1%87%D0%B5%D0%BD%D0%B8%D1%8F_%D0%B8%D0%B1

До начала создания систем информационной безопасности ряд отечественных
нормативных документов (ГОСТ Р ИСО/МЭК 15408 ГОСТ Р ИСО/МЭК 27000 ГОСТ Р
ИСО/МЭК 17799) и международных стандартов (ISO 27001/17799) прямо требуют
разработки основополагающих документов – Концепции и Политики информационной
безопасности. Если Концепция ИБ в общих чертах определяет, ЧТО необходимо
сделать для защиты информации, то Политика детализирует положения Концепции,
и говорит КАК, какими средствами и способами они должны быть реализованы.

\textbf{Концепция} (от лат. conceptio) --- генеральный замысел, определяющий
стратегию действий.

\textbf{Концепция защиты информации} --- это система взглядов на сущность,
цели, принципы и организацию защиты информации.

Концепция информационной безопасности используется для:
\begin{itemize}
  \item принятия обоснованных управленческих решений по разработке мер
      защиты информации;
  \item выработки комплекса организационно-технических и технологических
      мероприятий по выявлению угроз информационной безопасности и
      предотвращению последствий их реализации;
  \item координации деятельности подразделений по созданию, развитию и
      эксплуатации информационной системы с соблюдением требований
      обеспечения безопасности информации;
  \item для формирования и реализации единой политики в области обеспечения
      информационной безопасности.
\end{itemize}

Концепция защиты информации предполагает:
\begin{itemize}
  \item Определение понятия, сущности и целей защиты информации;
  \item Какую информацию необходимо защищать, каковы критерии отнесения ее к защищаемой;
  \item Дифференциацию защищаемой информации:
  \begin{description}
    \item[a)] по степеням конфиденциальности,
    \item[б)] по собственникам и владельцам;
  \end{description}
  \item Определение состава и классификации носителей защищаемой
      информации;
  \item Определение источников, видов и способов дестабилизирующего
      воздействия на информацию, причин, обстоятельств и условий
      воздействий, каналов, методов и средств несанкционированного доступа
      к информации;
  \item Определение методов и средств защиты информации;
  \item Кадровое обеспечение защиты информации.
\end{itemize}



\section{Задания}\label{sect3_b}
%
Используя предложенные образцы, разработать концепцию информационной
безопасности компании (\hyperref[var_PR1]{см. вариант}). При написании работы
можно ориентировать на план указанный в примере выполнения работы.

\begin{enumerate}
  \item Чтение материала по варианту.
  \item Структурное описание системы.
\end{enumerate}

Ниже приведен план для описания концепций ИБ объекта. В случае необходимости
могут быть внесены изменения:
\begin{enumerate}
	\item Общие положения

	Назначение Концепции по обеспечению информационной безопасности.
	\begin{enumerate}
	  \item Цели системы информационной безопасности
	  \item Задачи системы информационной безопасности.
	\end{enumerate}

	\item Проблемная ситуация в сфере информационной безопасности
	\begin{enumerate}
		\item Объекты информационной безопасности.
		\item Определение вероятного нарушителя.
		\item Описание особенностей (профиля) каждой из групп вероятных нарушителей.
		\item Основные виды угроз информационной безопасности Предприятия.
		\begin{itemize}
			\item Классификации угроз.
			\item Основные непреднамеренные искусственные угрозы.
			\item Основные преднамеренные искусственные угрозы.
		\end{itemize}
		
		\item Общестатистическая информация по искусственным нарушениям информационной безопасности.
		\item Оценка потенциального ущерба от реализации угрозы (см. Практическую работу № 1).
	\end{enumerate}
	
	\item Механизмы обеспечения информационной безопасности  Предприятия
	\begin{enumerate}
		\item Принципы, условия и требования к организации и функционированию системы информационной безопасности.
		\item Основные направления политики в сфере информационной безопасности.
		\item Планирование мероприятий по обеспечению информационной безопасности Предприятия.
		\item Критерии и показатели информационной безопасности Предприятия.
	\end{enumerate}
	
	\item Мероприятия по реализации мер информационной безопасности Предприятия
	\begin{enumerate}
		\item Организационное обеспечение информационной безопасности.
		\begin{itemize}
			\item Задачи организационного обеспечения информационной безопасности.
			\item Подразделения, занятые в обеспечении информационной безопасности.
			\item Взаимодействие подразделений, занятых в обеспечении информационной безопасности.
		\end{itemize}
		
		\item Техническое обеспечение информационной безопасности Предприятия.
		\begin{itemize}
			\item Общие положения.
			\item Защита информационных ресурсов от несанкционированного доступа.
			\item Средства комплексной защиты от потенциальных угроз.
			\item Обеспечение качества в системе безопасности.
			\item Принципы организации работ обслуживающего персонала.
		\end{itemize}
		
		\item Правовое обеспечение информационной безопасности Предприятия.
		\begin{itemize}
			\item Правовое обеспечение юридических отношений с работниками Предприятия .
			\item Правовое обеспечение юридических отношений с партнерами Предприятия.
			\item Правовое обеспечение применения электронной цифровой подписи.
		\end{itemize}
		\item Оценивание эффективности системы информационной безопасности Предприятия.
	\end{enumerate}
	
	\item Программа создания системы информационной безопасности Предприятия.
\end{enumerate}

\section{Пример выполнения работы}\label{sect3_c}
%

\section{Вопросы для контроля}\label{sect3_e}
%
