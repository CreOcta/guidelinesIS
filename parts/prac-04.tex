\chapter{Построение концепции информационной безопасности предприятия} \label{chapt4}%
\textbf{Мета роботи:~}%
Знакомство с основными принципами построения концепции ИБ предприятия, с
учётом особенностей его информационной инфраструктуры.
\section{Теоретические ведомости} \label{sect3_a}
% lab4.pdf
%https://habrahabr.ru/company/trinion/blog/322832/
%mu_ib.doc Практическая работа № 3
%https://studfiles.net/preview/5852002/page:2/
% http://securitypolicy.ru/%D1%88%D0%B0%D0%B1%D0%BB%D0%BE%D0%BD%D1%8B/%D0%BA%D0%BE%D0%BD%D1%86%D0%B5%D0%BF%D1%86%D0%B8%D1%8F_%D0%BE%D0%B1%D0%B5%D1%81%D0%BF%D0%B5%D1%87%D0%B5%D0%BD%D0%B8%D1%8F_%D0%B8%D0%B1

До начала создания систем информационной безопасности ряд отечественных
нормативных документов (ГОСТ Р ИСО/МЭК 15408 ГОСТ Р ИСО/МЭК 27000 ГОСТ Р
ИСО/МЭК 17799) и международных стандартов (ISO 27001/17799) прямо требуют
разработки основополагающих документов – Концепции и Политики информационной
безопасности. Если Концепция ИБ в общих чертах определяет, ЧТО необходимо
сделать для защиты информации, то Политика детализирует положения Концепции,
и говорит КАК, какими средствами и способами они должны быть реализованы.

\textbf{Концепция} (от лат. conceptio) --- генеральный замысел, определяющий
стратегию действий.

\textbf{Концепция защиты информации} --- это система взглядов на сущность,
цели, принципы и организацию защиты информации.

Концепция информационной безопасности используется для:
\begin{itemize}
  \item принятия обоснованных управленческих решений по разработке мер
      защиты информации;
  \item выработки комплекса организационно-технических и технологических
      мероприятий по выявлению угроз информационной безопасности и
      предотвращению последствий их реализации;
  \item координации деятельности подразделений по созданию, развитию и
      эксплуатации информационной системы с соблюдением требований
      обеспечения безопасности информации;
  \item для формирования и реализации единой политики в области обеспечения
      информационной безопасности.
\end{itemize}

%Концепция защиты информации предполагает:
%\begin{itemize}
 % \item Определение понятия, сущности и целей защиты информации;
%  \item Какую информацию необходимо защищать, каковы критерии отнесения её к защищаемой;
%  \item Дифференциацию защищаемой информации:
%  \begin{description}
%    \item[a)] по степеням конфиденциальности,
%    \item[б)] по собственникам и владельцам;
%  \end{description}
%  \item Определение состава и классификации носителей защищаемой
%      информации;
%  \item Определение источников, видов и способов дестабилизирующего
%      воздействия на информацию, причин, обстоятельств и условий
%      воздействий, каналов, методов и средств несанкционированного доступа
%      к информации;
%  \item Определение методов и средств защиты информации;
%  \item Кадровое обеспечение защиты информации.
%\end{itemize}

Концептуальная модель отвечает на общие вопросы и отражает схематично общую структуру модели информационной безопасности, на которой как на стержне строятся остальные модели и концепции информационной безопасности.

Для построения концептуальной модели информационной безопасности не зависимо от того насколько простая или сложная у Вас информационная система, необходимо как минимум ответить на три вопроса:
\begin{itemize}
  \item Что защищать?
  \item От кого защищать?
  \item Как защищать?
\end{itemize}
Это обязательный минимум, которого может быть достаточно для небольших информационных систем. Однако принимая во внимание возможные последствия, то лучше выполнить построение полной концептуальной модель информационной безопасности, в которой необходимо определить:
\\

\begin{minipage}{0.45\textwidth}
  \begin{enumerate}
    \item Источники информации:
      \begin{itemize}
    	\item документы;
    	\item средства связи;
    	\item сотрудники;
    	\item электронные носители.
      \end{itemize}
    \item Cтепень важности информации.
    \item Источники угроз:
      \begin{itemize}
    	\item внутренние;
    	\item внешние.
      \end{itemize}
    \item Цели угроз:
      \begin{itemize}
    	\item ознакомление;
    	\item дублирование;
    	\item модифицирование;
    	\item уничтожение.
      \end{itemize}
    \item Угрозы:
      \begin{itemize}
    	\item доступность;
    	\item целостность;
    	\item конфиденциальность.
      \end{itemize}
  \end{enumerate}
\end{minipage}
\hfill
\begin{minipage}{0.45\textwidth}
  \begin{enumerate}
    \setcounter{notes}{5}
    \item Способы доступа:
      \begin{itemize}
    	\item разглашение;
    	\item утечка;
    	\item несанкционированный доступ.
      \end{itemize}
    \item Направления защиты:
      \begin{itemize}
    	\item правовое;
    	\item организационное;
    	\item инженерно-техническое.
      \end{itemize}
    \item Средства защиты:
      \begin{itemize}
    	\item физические;
    	\item аппаратные;
    	\item программные;
    	\item криптографические.
      \end{itemize}
    \item Методы защиты:
      \begin{itemize}
    	\item упреждение;
    	\item предотвращение;
    	\item пресечение;
    	\item противодействие.
      \end{itemize}
  \end{enumerate}
\end{minipage}

\section{Задания}\label{sect3_b}
%
Используя предложенные образцы, определить концептуальную модель
безопасности компании.

\begin{minipage}{0.45\textwidth}
  \begin{enumerate}
    \item Отделение коммерческого банка;
    \item Поликлиника;
    \item Колледж;
    \item Офис страховой компании;
    \item Рекрутинговое агентство;
    \item Интернет-магазин;
    \item Центр оказания гос. услуг;
    \item Отделение полиции;
    \item Аудиторская компания;
    \item Дизайнерская фирма;
    \item Офис интернет-провайдера;
    \item Офис адвоката;
    \item Компания по разработке ПО;
    \item Агентство недвижимости;
    \item Туристическое агентство;
  \end{enumerate}
\end{minipage}
\hfill
\begin{minipage}{0.45\textwidth}
  \begin{enumerate}
    \setcounter{enumi}{15}
    \item Офис благотворительного фонда;
    \item Издательство;
    \item Консалтинговая фирма;
    \item Рекламное агентство;
    \item Отделение налоговой службы;
    \item Офис нотариуса;
    \item Бюро перевода (документов);
    \item Научно проектное предприятие;
    \item Брачное агентство;
    \item Редакция газеты;
    \item Гостиница;
    \item Праздничное агентство;
    \item Городской архив;
    \item Диспетчерская служба такси;
    \item Железнодорожная касса.
  \end{enumerate}
\end{minipage}

\section{Пример выполнения работы}\label{sect3_c}


\section{Вопросы для контроля}\label{sect3_e}
%
