\chapter{Семинар по теме стандартов в ИБ} \label{chapt5}%
\textbf{Мета:~}%
ознакомиться с основными положениями стандартов по обеспечению информационной
безопасности в распределённых вычислительных сетях.
\section{Требования к знаниям и умениям} \label{sect4_a}
%
\noindent Студент должен знать:
\begin{itemize}
  \item основное содержание стандартов по информационной безопасности
      распределённых систем;
  \item основные сервисы безопасности в вычислительных сетях;
  \item наиболее эффективные механизмы безопасности;
  \item задачи администрирования средств безопасности.
\end{itemize}

\noindent Студент должен уметь:
\begin{itemize}
  \item выбирать механизмы безопасности для защиты распределенных систем.
\end{itemize}

\section{Термины}
% http://cp11.megagroup.ru/d/871035/d/trufanovd.o.osnovyservisabezopasnosti.chast1.2014..pdf
\textbf{Распределённая информационная система}~---~совокупность аппаратных и
программных средств, используемых для накопления, хранения, обработки,
передачи информации между территориально удалёнными пользователями.

\textbf{Сервис \emph{(Сервисная деятельность)}}~--- это вид деятельности,
направленный на удовлетворение потребностей социальных субъектов посредством
оказания услуг.



 \textbf{Сервис безопасности}~---~это деятельность государственных
и частных организаций, а также отдельных специалистов, направленная на
удовлетворение потребностей социальных субъектов в безопасности.

\textbf{Цель сервиса безопасности} − удовлетворение потребностей в
безопасности индивидуальных и групповых социальных субъектов.
\textbf{Сущность сервиса безопасности} состоит в оказании услуг, направленных
на обеспечение безопасности. \textbf{Услуга безопасности} – это деятельность
субъекта безопасности, направленная на удовлетворение потребности заказчика в
безопасности, а также результат взаимодействия исполнителя и заказчика услуги
безопасности, выраженный в виде полезного эффекта.

\section{Темы для обсуждения}\label{sect4_b}
\begin{enumerate}
  \item Механизмы безопасности.
  \item Сервисы безопасности в вычислительных сетях.
  \item Функций и механизмов безопасности.
  \item Администрирование средств безопасности.
  \item Международные стандарты.
  \item Стандарты ГОСТ и ДСТУ.
\end{enumerate}

Ссылки на литературу:
1.~Щербаков А. Ю. Введение в теорию и практику компьютерной безопасности. – М.: Издательство Молгачева С. В., 2001.%
2.~Теория и практика обеспечения информационной безопасности / Под ред. П. Д. Зегжды. – М: Яхтсмен, 1996.%
3.~Галатенко В. А. Основы информационной безопасности. – М: Интернет-Университет Информационных Технологий – ИНТУИТ. РУ, 2003.%
4.~Галатенко В. А. Стандарты информационной безопасности. – М: Интернет-Университет Информационных Технологий – ИНТУИТ. РУ, 2004.%
5.~www.iso.ch – Web-сервер Международной организации по стандартизации.%
