\chapter{Cisco. Топология сетей} \label{chapt3}%
\textbf{Мета роботи:~}%
Напомнить основные концепции сети, их связи, уязвимости  и основные узлы.
\section{Теоретические ведомости} \label{sect3_a}
%
\paragraph{Информационная среда}

Сеть со стороны ИБ.

Основые <<дыры>> в сети

\paragraph{Пути несанкционированного доступа}


\subparagraph{Средства защиты информации}

(всяко-разно)%
\href{https://ru.wikipedia.org/wiki/%D0%97%D0%B0%D1%89%D0%B8%D1%82%D0%B0_%D0%B8%D0%BD%D1%84%D0%BE%D1%80%D0%BC%D0%B0%D1%86%D0%B8%D0%B8_%D0%B2_%D0%BB%D0%BE%D0%BA%D0%B0%D0%BB%D1%8C%D0%BD%D1%8B%D1%85_%D1%81%D0%B5%D1%82%D1%8F%D1%85}%
{Ссылка на вики} по этой теме

 -- Маршрутизатор как способ защиты. (обязательно включить)
    - Фильтрование сети

    - экранирование сети

Защита информации в сети Интернет:

\section{Задания}\label{sect3_b}
%
\noindent Уровень 1.

\begin{enumerate}
  \item Изучить предложенную топологию в Cisco. Представлена на \todo{рис.}
  \item Изменить модель по варианту.
    \begin{enumerate}
    \item Построить реализацию заданной сети в Cisco или другом ПО.
    \item Указать недостатки заданной системы.
    \end{enumerate}
\end{enumerate}

\noindent Уровень 2.

\begin{enumerate}

  \item Настроить маршрутизатор по варианту.
    \begin{enumerate}
        \item Шаг 1.
        \item Шаг 2.
        \item Шаг 3.
        \item Шаг 4.
        \item Шаг 5.
        \item Шаг 6.
        \item Шаг 7.
        \item Шаг 8.
    \end{enumerate}
  \item Разделение подсети, \todo{протокол RIP}
\end{enumerate}
\section{Пример выполнения работы}\label{sect3_c}
%
\section{Варианты}\label{sect3_d}
%
\section{Вопросы для контроля}\label{sect3_e}
%
\begin{enumerate}
  \item Какие есть средства защиты частной сети?
  \item Что такое шлюз сетевого уровня?
  \item Преимущества и недостатки использования сетевых экранов.
  \item Что такое списки доступа? Каковы их цели и применение?
  \item Приведите пример списка доступа.
  \item Что такое маска подсети?
  \item Каков принцип маски и как происходит фильтрация
      пакетов
\end{enumerate}
