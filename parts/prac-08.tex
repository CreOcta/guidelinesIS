\chapter{Анализ рисков} \label{chapt8}%
\textbf{Мета роботи:~}%
Изучение анализа рисков. Формирование навыка определения угроз и защита.
\section{Теоретические ведомости} \label{sect7_a}
% https://www.intuit.ru/studies/courses/531/387/lecture/8992?page=1
\paragraph{Риски в информационной безопасности}

\paragraph{Анализ стойкости системы}

\paragraph{Правила определения угроз и защиты информации}


\textbf{Риск ИБ} --- потенциальная возможность использования определенной
угрозой уязвимостей актива или группы активов для причинения вреда
организации.

\textbf{Уязвимость} --- слабость в системе защиты, делающая возможной
реализацию угрозы.

\textbf{Угроза ИБ} --- совокупность условий и факторов, которые могут стать
причиной нарушений целостности, доступности, конфиденциальности информации.

\textbf{Информационный актив} --- это  материальный или нематериальный
объект, который:
\begin{itemize}
  \item является информацией или содержит информацию,
  \item служит для обработки,  хранения или передачи информации,
  \item имеет ценность для организации.
\end{itemize}






\section{Задания}\label{sect7_b}
%
\begin{enumerate}
  \item Защита объекта по варианту из \todo{табл.}.
  \item Оценка качества защиты.
\end{enumerate}
\section{Пример выполнения работы}\label{sect7_c}
%
\section{Варианты}\label{sect7_d}
%
