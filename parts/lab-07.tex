\chapter{Пассивный анализ данных} \label{chapt8}%
\textbf{Мета роботи:~}%
Применение методов и технологий испытания программного и аппаратного уровней
комплексной защиты информации для проведения атаки на КИС с целью
установления уязвимостей.
\section{Теоретические ведомости} \label{sect7_a}
% https://informationsecurityweb.wordpress.com/2016/05/27/%D0%B0%D0%BD%D0%B0%D0%BB%D0%B8%D0%B7-%D1%83%D0%B3%D1%80%D0%BE%D0%B7-%D0%B8%D0%BD%D1%84%D0%BE%D1%80%D0%BC%D0%B0%D1%86%D0%B8%D0%BE%D0%BD%D0%BD%D0%BE%D0%B9-%D0%B1%D0%B5%D0%B7%D0%BE%D0%BF%D0%B0/

Анализ трафика и сбор критичной информации программами пассивного анализа
является одним из методов получения критичной информации о корпоративной
информационной системе. Для реализации необходимо иметь специализированное
программное обеспечение.

В сфере обеспечения ИБ необходимо обнаруживать и нейтрализовать вредоносный
код прежде, чем он нанесет ущерб информационной системе. Проанализировав
нестандартное «поведение» трафика, можно определить его как угрожающее и,
блокировав действие соответствующих программ, сообщить пользователю об
инциденте.

Проведение анализа трафика и сбор критичной информации с применением программ
пассивного анализа (программ-снифферов и программ обнаружения вторжений)
включает:

\begin{itemize}
  \item определение слабых мест в защите сервисов: FТP, TFТP, SSH, Finger,
      HTTР, IMAP SMTP, NetBIOS/SMB, RPC;
  \item выявление слабых мест сетевых информационных служб (NIS);
  \item проверка на возможность IР-спуфинга;
  \item проверка маршрутизации из источника rlogin, rsh и telnet;
  \item проверка IР-переадресации (forwarding);
  \item проверка сетевых масок и временных меток (timestamp)  ICMP;
  \item проверка инкапсуляции пакета MBONE;
  \item проверка инкапсуляции APPLEТALK IP, IPX, Х.25, FR;
  \item проверки резервированных разрядов и паритет-протоколов;
  \item проверка специализированных фильтров;
  \item проверка фильтров с возможностью нулевой длины ТСР и IP;
  \item проверка на передачу сверхнормативных пакетов;
  \item проверка опций post-EOL для ТСР и IP;
  \item проверка  наличия в Web-сервисах  уязвимых  сценариев, на базе
      Basic, Script, JavaScript, Perl и ActiveXo;
  \item проверка программного обеспечения на закрытие всех известных уязви-
      мостей данной платформы.

\end{itemize}


\section{Задания}\label{sect7_b}
%
\begin{enumerate}
  \item Выбрать задание по варианту из \todo{табл.}
  \item Провести анализ интернет трафика.
  \item Провести выборку на массиве данных.
  \item Запустить \todo{фишинговую угрозу}.
  \item Собрать дополнительные данные и найти вирус с помощью данного
      метода.
  \item Отобразить результаты в отчёте.
\end{enumerate}
\section{Пример выполнения работы}\label{sect7_c}
%
\section{Варианты}\label{sect7_d}
%
\section{Вопросы для контроля}\label{sect7_e}
%
\begin{enumerate}
  \item Что такое анализ данных в ИБ?
  \item Какие бывают методы анализа?
  \item Какое ПО для анализа данных вы знаете?
  \item Как работает метод пассивного анализа?
  \item Как работает метод активного анализа?
  \item Какова надёжность методов анализа данных?
  \item Какие особенности, достоинства и недостатки анализа вы знаете?
\end{enumerate}
