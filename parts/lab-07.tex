\chapter{Пассивный анализ данных} \label{chapt8}%
\textbf{Мета роботи:~}%
Изучить способы анализа данных для дальнейшего применения. Применение пакета
данных в целях защитных и атакующих систем.
\section{Теоретические ведомости} \label{sect7_a}
% https://informationsecurityweb.wordpress.com/2016/05/27/%D0%B0%D0%BD%D0%B0%D0%BB%D0%B8%D0%B7-%D1%83%D0%B3%D1%80%D0%BE%D0%B7-%D0%B8%D0%BD%D1%84%D0%BE%D1%80%D0%BC%D0%B0%D1%86%D0%B8%D0%BE%D0%BD%D0%BD%D0%BE%D0%B9-%D0%B1%D0%B5%D0%B7%D0%BE%D0%BF%D0%B0/

Зачем нужен анализ данных в ИБ.

\paragraph{Виды анализа данных}

\paragraph{Выбор средств для пассивного анализа}

Выборка и поиск атак на ресурс используя методы пасивного анализа.




\section{Задания}\label{sect7_b}
%
\begin{enumerate}
  \item Выбрать задание по варианту из \todo{табл.}
  \item Провести пассивный анализ интернет трафика.
  \item Провести выборку на массиве данных.
  \item Запустить \todo{фишинговую угрозу}.
  \item Собрать дополнительные данные и найти вирус с помощью данного
      метода.
  \item Отобразить результаты в отчёте.
\end{enumerate}
\section{Пример выполнения работы}\label{sect7_c}
%
\section{Варианты}\label{sect7_d}
%
\section{Вопросы для контроля}\label{sect7_e}
%
\begin{enumerate}
  \item Что такое анализ данных в ИБ?
  \item Какие бывают методы анализа?
  \item Какое ПО для анализа данных вы знаете?
  \item Как работает метод пассивного анализа?
  \item Как работает метод активного анализа?
  \item Какова надёжность методов анализа данных?
  \item Какие особенности, достоинства и недостатки анализа вы знаете?
\end{enumerate}
